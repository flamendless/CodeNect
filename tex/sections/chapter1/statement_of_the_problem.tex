\flushleft
\textbf{Statement of the Problem}\\

\justifying
\parx
The fundamental concepts of programming are essential basics that are necessary
for programmers to master. Concepts such as syntax, semantics, variables, data
types, data structures, logic, conditionals, loops, algorithm, and memory are
key to easily understanding and getting better at programming as it is a
discipline (\cite{prahofer_hurnaus_wirth_mossenbock_2007}).
Programming is a skill which can be boring, intimidating, and unrelated to daily
activities and experience. Students are lacking in understanding of the
execution of a program (\cite{tan_2019}).
Programming education requires the assistance of technology itself through
software in improving the quality of learning. The traditional method of pure
lecture is nowadays complimented with the application of softwares. But most
tools are not beginner-friendly and are cluttered with features that present
confusion and steep learning curve in familiarity and mastery that diminish the
learning experience (\cite{tsukamoto_2016}).

\justifying
\parx
The assessment of the respondents under the courses with programming subjects
(See Appendix Figure \ref{dataresults1}) shows that students (See Appendix
Figure \ref{dataresults13})
are not familiar and not well versed on fundamental concepts and find it
difficult to understand
(See Appendix Figure \ref{dataresults12} and \ref{fishbone1}).

\parx
Basic concepts such as loops, memory management, and functions are easily
understood individually, but combining them into a program has confused students
(See Appendix Figure \ref{fishbone2}). Respondents failed to correctly
answer the assessment (See Appendix Figure \ref{dataresults13}).

\parx
Survey shows that 76\% of students use outdated text-based editors in their
laboratory classes such as Notepad++, DevC++, and TurboC/C++ (See
Appendix Figure \ref{dataresults9}), while only 24\% use professional and modern editors
for programming. This traditional text-based editors are general tools and are
not oriented for learning of beginners and thus not effective (See Appendix
Figure \ref{fishbone3}).
