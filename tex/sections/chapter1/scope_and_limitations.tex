\flushleft
\textbf{Scope and Limitations of the Study}\\
\justifying
\parx
The study focuses on the development of a CodeNect: Visual Programming Software
for learning the fundamentals of programming. The software will prioritize
simple and basic functionalities over numerous features for the purpose of
learning and education.

The software is to be developed in Linux operating system using the C++ programming
language, Vim as code editor, terminal for building, OpenGL and GLFW framework for
rendering and input.
The source is to be released as open-source with appropriate license to improve
contributions. Since the software is stand-alone desktop program, there will be no
account management. The software is only accessible by a single user such as the
student or instructor.
The software fully works in offline mode.

The software is designed with seven core modules: Visual Nodes Module,
Filesystem Module, Input and Output Module, Debug Module, Simulation Module,
Transpiler Module, and Assessment Module.

\flushleft
\textbf{\textit{Visual Nodes Module}}\\
\justifying
\parx
Nodes are graphical elements that serve as the building blocks of the software.
Nodes can be used as a variable, logic, and conditionals. The properties of the node
are position, size, and type. The fields of the node that are visible to the
users, which can be modified, are name, and value(s). Each node has input
socket(s) and output pins which are used for the flow of logic and redirection
of data. The visibility of the sockets and pins of a node is dependent on its
type.  For example, nodes that are constant variables will only enable the
output socket as it is read-only, while regular variable nodes allow for both
sockets.  Nodes are connected to one another through the use of wires. This
relation of nodes is called the node graph. The flow of logic is easily
determined using the wires with directional arrows signifying the direction of
the logic.

\flushleft
\textbf{\textit{Filesystem Module}}\\
\justifying
\parx
This module serves as the interface between the software and the user's machine for
handling files. The module have four main functionalities, creation, modification,
reading of files, and backup. One feature of the software that benefits from this
module is the importation of exercises from package format file which allows for
more learning materials that greatly increases the possible usage of the software.

\parx
When a user starts a new project, a template project structure with base files are
created by the module and is saved into the user's machine. The modules assure that
the project is stored with proper permission and in safe location. Modification such
as addition or deletion is also handled by the module. The reading of files and
project functionality takes into consideration the validity and safeness of the file
and handles if the file is corrupted. The backup functionality regularly makes
a backup of file in case of emergency such as program crash or user-side accident.

\flushleft
\textbf{\textit{Input and Output Module}}\\
\justifying
\parx
This module captures user input events such as key press, mouse movement, mouse
click, and so on. The module is responsible for processing and responding events and
performing actions based on the event. This ensures that the interaction between the
user and the software provides rich experience in terms of usability and learning.

\parx
This module handles output to the user. File, displays, views, and screens are examples
of the possible types of output. The module manages everything that is rendered to the
screen for the user to see such as the elements, the visual graph, a combination of the
visual nodes connected to each other, the assessment files or reports, and the
simulation view.

\flushleft
\textbf{\textit{Debug Module}}\\
\justifying
\parx
This module will linter and give feedback and indication to the user whenever there
is an attempt to perform an action that is faulty in logic. For example, the red
color means error or danger while the yellow color means warning. The color based
feedback and highlight is used in combination with useful messages giving more
detailed information regarding the fault. These are placed accordingly to the source
of the fault, whether in the node or in the wire. Runtime errors or warnings during
the simulation stage is propagated to the user with detailed information and
explanations about the probable cause of error and displays tips in debugging and
fixing the program.

\flushleft
\textbf{\textit{Simulation Module}}\\
\justifying
\parx
The process of simulation involves the compiling, building, and running the visual
code is executed by this module. The compilation stage involves going from the main
node which is the entry point of the program followed by the importing of libraries
and packages (depending on the target language).  After that is the declaration of
variables and methods and will continue to parse and convert the visual code to its
equivalent source code.

\parx
The compilation stage is followed by the building, also known as linking stage.
During this stage, the compiled source code is linked with the necessary libraries
required by the target programming language. Examples of this are Dynamic Linking
Libraries (.dll) and Shared Objects (.so) files.

\parx
After the compilation and building stage is the simulation or execution stage.  This
executes the program and run it with additional features enabled to allow a more
dynamic and intuitive interaction between the user and the program.

\flushleft
\textbf{\textit{Transpiler Module}}\\
\justifying
\parx
This module transpiles the visual code made by the user into source code in target
programming language. This module tests that the transpiled source code compiles
and runs correctly as well. This allows for learners to see and compare their
work into other programming languages which is part of the education curriculum.
This serves as helper for their transition from learning the fundamentals of
programming into its application towards programming languages that are more
robust, high quality, and industry-standard. The module will walk through each
nodes provided by the Visual Nodes Module and will parse each to create a tree
structure that will be read and converted by the module into the selected
programming language by the user.  Since this module is reponsible for the
conversion of each visual data into their corresponding textual data, this
module also evaluates the parsed tree structure in order to check and interpret
expressions in programming.

\flushleft
\textbf{\textit{Assessment Module}}\\
\justifying
\parx
The functionality of providing exercises designed for the learning of topics and
concepts in programming and evaluation of the results are handled by the Assessment
Module. The possible types of exercise range from output-based program to writing
an algorithm that has memory and time limitation and complexity. The module can
store the evaluated performances or grades of the user for further analyzation
and can provide basic reports such as determining what concepts do most students
find it difficult to learn within a set of time.

\parx
The software is limited to simulating text-based or command/terminal prompts as the
priority is learning the fundamentals of programming. The software does not also
compete as an Integrated Development Environment (IDE). The software does not provide
networking functionalities such as connection to the internet to send or fetch data.
There is also no access level or account management for the user.\\
