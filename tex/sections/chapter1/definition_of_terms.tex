\flushleft
\textbf{Definition of Terms}\\
\justifying

\parx
\textbf{Algorithm} is a set of instructions designed to perform a specific task.
Assessments provide exercises for understanding algorithm as a fundamental concept.

\parx
\textbf{Bug} is unwanted behavior caused by faulty logic. Learning fundamental
concepts reduces proneness to bugs.

\parx
\textbf{Building} is linking compiled code with libraries to make an executable.
This is a key step in the simulation of the system.

\parx
\textbf{CodeNect} combination of the word "Code" which is synonymous to
programming, and "Nect" which is taken from "Connect", based on the connection
of multiple nodes to create a program.

\parx
\textbf{Compilation} is turning human-readable language to machine language.
This is the initial step in the simulation of the system.

\parx
\textbf{Conditionals} are statements or expressions comparing logic in
programming.  Assessments provide exercises for understanding conditionals as a
basic and fundamental concept.

\parx
\textbf{Data} is information digitally stored in and processed by computer.
Data are stored in and represented by visual nodes.

\parx
\textbf{Data Structure} is grouping and storage of data efficiently in memory.
Assessments provide exercises for understanding data structures as a fundamental concept.

\parx
\textbf{Data Types} are attributes to determine the size and type of data.
Visual nodes represent data types by color for categorization.
Assessments provide exercises for understanding data types as a fundamental concept.

\parx
\textbf{Debugging} is the process of finding and resolving of bugs in a program.
The system provides interactive and visual way in learning debugging.

\parx
\textbf{Dynamic Linking Libraries} contain functions and other information used
by a Windows program. These are necessary for the system to be stand-alone.

\parx
\textbf{Exercise} is a simple coding problem statement with expected output.
An assessment is in the form exercise.

\parx
\textbf{Graphical User Interface} is the interface of interactive graphical elements.
The system provides clean and minimal GUI for easier usage and learning.

\parx
\textbf{Integrated Development Editor} provides features for easier text programming.
This has features not necessary for the purpose of learning and as such the system
is not an IDE.

\parx
\textbf{Loops} are statements or expressions that repeat a sequence of code.
Assessments provide exercises for understanding loops as a fundamental concept.

\parx
\textbf{Memory} is data storage of computer allotted for a program to use.
Assessments provide exercises for understanding memory as a fundamental concept.

\parx
\textbf{Module} is a component in software that provides specific functionalities.
The system is composed of multiple modules that serve and provide the core.

\parx
\textbf{Nodes} are visual elements that contains data and can be linked with other nodes.

\parx
\textbf{Programming} is the writing of code for instructing computers what to do.
The system prioritizes users in learning quality programming.

\parx
\textbf{Programming Language} is a human-readable language that a programmer uses.
The system provides other programming languages for output.

\parx
\textbf{Runtime Error} is an error that occurs when the program is running.
Learning fundamentals of programming will reduce the occurences of runtime error.

\parx
\textbf{Semantics} is the evaluation of syntax and tokens of a programming language.
Learners are not restricted by learning semantics in visual programming language.

\parx
\textbf{Software} is a program or collection of instructions operating the computer.

\parx
\textbf{Socket} is the input/output slot for connecting nodes to form a graph.

\parx
\textbf{Syntax} is a set of rules that defines the structure of symbols.
Its version in visual programming are nodes and connectors.

\parx
\textbf{Technology} is the application of knowledge in a particular area or field.
The system applies technology to provide richer learning experience.

\parx
\textbf{Text-based Programming} is the use of texts to write a program.
This is also an output by the visual code when transpiled.

\parx
\textbf{Terminal} is an interface that accepts input or command in text form.
Users can interact and test their program through the terminal.

\parx
\textbf{Transpilation} is the conversion of code to other programming language.
The system integrates this to facilitate learning and easier adoption to real
programming languages.

\parx
\textbf{User-Interface} is the layer that the user controls and interacts with.
This spans the concepts of text user-interface and graphical user-interface.

\parx
\textbf{Variable} a named reference that holds a value in memory for the user to use.
Assessments provide exercises for understanding variable as a fundamental concept.
