\justifying
\parx
As the innovation in technology is continuously making its progress in improving
the quality of life. With this nature of technology comes the essential need for
programming skill as core proficiency. The competition in the field that
developers and programmers alike strive for is becoming harder to get into due
to high standards and requirements. One of the requirements for a programmer and
developer is to have expertise in technical skills that include multiple
programming languages (\cite{tsai_yang_chang_2015}). Without the proper
knowledge and understanding in programming in its fundamental level and depth,
one will find it difficult to adapt to the constantly shifting world of
computer.

\parx
Improving learning without prolonging the time allotted in each
academic year needs to focus on enhancing the properties of the software that
are both utilized as teaching and learning tools by the instructor and the
student. A system that is implemented using modern tools, industry
standard design, and functionality that focuses in simplicity, readability, and
learning experience. Modern technology increases the rate of knowledge
acquisition and absorption through its usage and implementation in education
(\cite{raja_nagasubramani_2018}). The advancement in technology greatly
contributes to education as it enables convenience in communication and
presentation of knowledge and information almost instantaneously
(\cite{anggrawan_ibrahim_m}). A software that prioritize functionality over
unnecessary features to ensure that the user is not overloaded with information
in the screen that is unnecessary. Users perceive numerous features in a product
to be useful and engaging but such can result in fatigue
(\cite{thompson_hamilton_rust_2005}).

\parx
Software with the necessary tools and functionalities oriented towards learning
purposes and is also designed to and packaged with coding exercises and problems
which range from beginner, intermediate, to advance difficulties is not popular
and lacking in availability. Features that are carefully selected and designed
towards showing and comparing various solutions that are working in the context
that they meet the requirements and output and are technically correct, but not
all will meet the standard when it comes to better quality which is the
advantages in skills acquired when mastering the fundamentals of programming.
This approach in problem solving allows learners to develop logical and critical
thinking through the application of the theory of variation, wherein some
aspects that are critical must vary while other aspects stay constant
(\cite{cheng_2016}). This is effective in the domain of programming as even a
slight change in data amounts to a change in effect and output.

\parx
Programming is a skill wherein it focuses in the connection of logic rather than
memorizing information as that of in other domain, the curve in starting to
learn it is steeper compared to actually applying it in real works and mastering
it. Mastering a programming language is not an easy task, but in general all
share common concepts. Learning by heart these core concepts and fundamental
knowledge will help programmers to easily learn and master any existing or new
programming language through the reiteration of principles that all programming
languages are built and modeled upon. For the design and decision that go behind
the creation of new programming languages are reevaluation of existing studies,
syntax, semantics, and inspired by widely used and accepted languages
(\cite{stefik_siebert_slattery_stefik_2011}). For anyone who is new to
programming, the topics can be a daunting and intimidating task. Failure in
familiarization and application in the early academic years and progressing to
the next period wherein advance subjects are covered bereaves the overall
learning of the student.

\parx
Visual programming enables learning through interaction and manipulation of
graphical elements which can provide more feedback through the use of color,
size, and icons. The abstract and high-level concepts are represented visually
that can facilitate learners by variable observation, logic flow tracing, and
debugging skills (\cite{tsai_yang_chang_2015}).
The use of visual programming language has been effective in facilitating and
assisting beginners in programming as visual programming lessens the time
needed to learn new topics and concepts in advance subjects. There are fewer
difficulties in learning as well as visualization provides higher cognitive
levels of understanding for most concepts (\cite{armoni_2015}).

% \parx
% The integration of a visual nodes and graphs as the interface in programming instead
% of the traditional text-based language is simpler, more appealing to use, and more
% intuitive to get started with. Visual programming saves time in specifying the code
% textually and manually input. It enables learning through interaction and manipulation
% of graphical elements which can provide more feedback through the use of color, size,
% and icons. The abstract and high-level concepts are represented visually that can
% facilitate learners by variable observation, logic flow tracing, and debugging skills
% (\cite{tsai_yang_chang_2015}). This visual programming tool can also export an
% equivalent source code to other programming languages that are in the curriculum
% such as C/C++, Java, and Python. The tandem of visual code and textual code allows
% programmers to write programs while learning fundamental concepts as seeing the code
% in both formats. This allows to ease the transition from introductory visual programming
% skill to real text-based programming language. (\cite{alam_bush_2016})
