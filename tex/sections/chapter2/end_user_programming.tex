% \addcontentsline{toc}{subsection}{End-User Programming Approaches}
\flushleft
\textbf{End-User Programming Approaches}\\
\justifying

\parx
The types of programmer range from professional, novice, and end-user. Professional
programmers are whole main work is to develop or maintain a code base. Novice
programmers can be thought of as professional programmers under training. End-user
programmers are those that program but programming is not their main function or
career. Another case for comparing the types of programmers are their interest and
knowledge when it comes to programming itself. Professional and novice programmers has
more in-depth understanding about the processes involved in programming and are capable
of programming using traditional semantic and text-based code whereas end-user
programmers do not. The following are the various approaches to programming for
end-users.

\flushleft
\textbf{Preferences Programming}\\
\justifying
\parx
This is provided by applications to allow the users to modify the behaviors and visual
appearances of the application itself. These are predefined options in the form of
checkbox, radio button, or dropdown menu that the user can interact with to suit
their preferences.

\flushleft
\textbf{Programming by Demonstration}\\
\justifying
\parx
This programming approach uses a system for recording user inputs for future playback.
This allows users to work in a general way to program the system what to accomplish
by showing the actual actions. This approach is tightly rule-based to enforce the
smooth replaying of actions (\cite{harrison_2004}).

\flushleft
\textbf{Spreadsheet Programming}\\
\justifying
\parx
This approach focuses on the requirements of mathematical knowledge and skills in
building formuals and models in the form of functions. Since this approach is
visual-based as the spreadsheets constantly indicate the result of calculations for
errors (\cite{abraham_burnett_erwig_2009}).

\flushleft
\textbf{Script Programming}\\
\justifying
\parx
This approach uses scripting languages as opposed to full programming language. A
scripting language can be a subset of a programming language or embedded language. These
languages are tiny and are generally designed to be used by people whose main domain
is not programming. This application can range from game design, music generator,
video effects, and prototyping (\cite{ousterhout_1998}).
