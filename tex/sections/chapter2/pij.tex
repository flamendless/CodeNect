% \fakesubsection{Environment pi J for Visual Programming in Java}
\flushleft
\textbf{Environment pi J for Visual Programming in Java}\\
\justifying

\parx
There is a wide known visual tools for programming in Java such as Symantec
Cafe, Visual J++ by Microsoft, and Visual Age for Java by IBM. The main feature
of the tools listed is the possibility to modify the graphical elements of the
user interface. The tools support representation in graphic forms of program
structure for packages, classes, methods, and properties. The tools apply the
visual representation to view, modify, and debug the programs. There is a
difficulty to consider that is related to using the tools listed, and that is it
is necessary for the users to have relatively high knowledge in programming,
particularly in the Java programming language.

\parx
Pi J is a programming environment for developing professional software basing on
the concepts of object-oriented programming. Pi J can operate with two kinds of
file format and it allows opening and saving of text file in plain Java. It also
allows opening and saving of file as structured format that the tool can read.
The tool supports most of the features provided by the Java programming
language (\cite{prokhorov_kosarev}).
