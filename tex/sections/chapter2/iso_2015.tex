% \addcontentsline{toc}{subsection}{ISO/IEC/IEEE 29119-4:2015
% Software and systems engineering — Software testing}
\flushleft
\textbf{ISO/IEC/IEEE 29119-4:2015 Software and systems engineering — Software
testing}\\
\justifying

\parx
The ISO/IEC/IEEE 29119 Software and Systems Engineering - Software Testing is a
a set of five standards for software testing internationally recognized and
approved. It was first developed in year 2007 and was released in year 2013.
This standard defines the following for usage with software development
lifecycle: vocabulary, processes, documentation, Techniques, and a process
assessment model for testing (\cite{iso_2015}).

\parx
The ISO/IEC/IEEE 29119 has the following standards: Concepts and Definitions,
Test processes, Test Documentation, Test Techniques, and Keyword-driven testing.

\parx
\textbf{Part 1.}
\justifying
It provides definitions, description of the concepts, and the application of the
definitions and concepts to the other parts of the standard. It introduces the
vocabulary on which the standard is built and provides an example of its
application in practice.

\parx
\textbf{Part 2.}
\justifying
Defines the common test process model for testing software intended for
organizational use. It follows the test descriptions, test management, and
dynamic levels at the organizational levels. It can be used in conjunction with
other standards.

\parx
\textbf{Part 3.}
\justifying
Deals with Documenting the software test processes and provides templates and
examples that are produced during the test process. It has the following
category: Organizational Test Process Documentation, Test Management Process
Documentation, and Dynamic Test Process Documentation.

\parx
\textbf{Part 4.}
\justifying
This part of the standard provides standard definitions of software test design
techniques, also known as test case design techniques or test methods, and the
coverage measures that are to be used during the design of tests and
implementation of the processes defined in previous part or other standard used
with conjunction. It has the following test design techniques:
Specification-based, Structure-based, and Experience-based.
