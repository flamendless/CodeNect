% \fakesubsection{Prototype of Visual Programming Environment for C Language Novice Programmer}
\flushleft
\textbf{Prototype of Visual Programming Environment for C Language Novice
Programmer}\\
\justifying

\parx
The C programming language is often the first programming language taught and
learned in higher education. This is also the case for the author's institution,
the Kanagawa Institute of Technology - Department of Information Engineering.
The C language classes are thrice a week in the first year. The students use
Visual Studio for programming. In text-based languages like C, students must
memorize tokens or keywords such as data type and syntax in general. These are
difficulties for beginners. Even a single typographical error in text-based
programming can lead to undesirable messages like compilation errors.

\parx
The research and development of a programming support system for beginner
programmers has attained an advanced stage. One of which is the block-based
visual programming language environment such as Scratch or Snap!. Block-based
programming environments allow student to edit the program by visually selecting
blocks and combining them together. They can create programs even without the
need to memorize keywords or to understand the syntax in formulating the grammar
for the semantics. Scratch and Snap! use their own programming languages which
are not suitable for learning other popular languages like C, C++, or Java.

\parx
With these factors considered, in this investigation the researchers have
developed a visual programming environment for the C language with the goal of
lowering the barriers between starting in programming and learning of the C
programming language. This programming environment is a Web application that has
functions to edit a C language program and to execute and step through the
program and trace the changes in the variables (\cite{abe_fukawa_tanaka_2019}).
