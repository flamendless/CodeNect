% \addcontentsline{toc}{subsection}{The Scratch Programming Language and
% Environment}
\flushleft
\textbf{The Scratch Programming Language and Environment}\\
\justifying

\parx
The Scratch is visual programming language and environment that allows users to
create media-rich projects such as games and interactive media. Scratch is an
application that is used to create projects provided with media and scripts.
Assets like images and audio can be imported or created directly within the
application through the built-in paint and audio recorder tools. Any of these
things, that a text-based programming environment can, is possible through the
use of colorful command blocks to control graphical objects called sprites set
in a background called the stage
(\cite{maloney_resnick_rusk_silverman_eastmond_2010}).

\parx
Users learn Scratch as they use it, experimenting commands from the provided
palette or exploring projects from existing projects. Scratch was designed to
allow scripting, provide immediate feedback for the execution of scripts, and
making the execution and data visible to motivate users for such self-directed
learning. The following are the design followed by Scratch:

\parx
\textbf{Single-Window user interface.}
The user interface of Scratch makes
navigation easier as it only uses a single window with multiple panes to ensure
that key components are always visible.

\parx
\textbf{Liveness and Tinkerability.}
One of the key features of Scratch is that it is always live. There is no
compilation phase. The program listens to each interaction between the user and
the interface and process it accordingly to provide smoother experience and
immediate feedback to the users.

\parx
\textbf{Making Execution Visible.}
Scratch provides feedback visually to show the execution of scripts. There is an
indicator around the script element for users to easily identify and follow it.
This feature helps users understand the flow of the program such as when it is
triggered and the duration of the script.

\parx
\textbf{No Error Messages.}
When people play with blocks such as LEGO, there is no error message
encountered. Blocks fit together only in certain ways, and it is easier to get
things right than wrong. The shapes of the block suggest the possible position
and orientation for it to properly fit and experimentation and experience teach
what works and what does not.

\parx
\textbf{Making Data Concrete}.
In most programming languages, variables are abstract and harder to understand.
Scratch turns variables into concrete objects that the user sees and
manipulates, through tinkering and observation making programming easier to
understand.

\parx
\textbf{Minimizing the Command Set}.
Scratch aims to minimize the number of command blocks while still allowing a
wide range of project types. One might point out that flexibility, convenience
of the programmers, and more features are better than a small set of commands.
In Scratch, every command consumes screen space in the command palettes, so
there is more cost to increasing the set of commands. Addition of more commands
requires additional categories or forcing the user to navigate and scroll down
to see all the commands available within a given group.
