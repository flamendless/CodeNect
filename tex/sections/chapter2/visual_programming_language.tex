% \addcontentsline{toc}{subsection}{Visual Programming Language}
\flushleft
\textbf{Visual Programming Language}\\
\justifying

\parx
Conveniently, explaining what a program does leads to the usage of graphical
representation of the control flow, connections, shapes, and more elements. This could
also be applied for programming and learning it. Visual programming languages enable
users to achieve the same concept (\cite{remi_2015}).

\parx
Aside from programming logic, visual programming languages are also used in a wide
variety of applications and has corresponding types. Some of these are the following:

\parx
The drag-and-drop type of visual programming language uses blocks as elements that can
fit into other blocks for composition, similar to a jigsaw puzzle piece. A study that
compared drag-and-drop visual programming to text-based programming concluded that
respondents were confident in their knowledge and skill in performing simple and basic
command with visual language programming, but found it harder to express what they want
in drag-and-drop for complex problems. This suffice to using visual programming as first
steps in basic programming before proceeding to complex concepts (\cite{disalvo_2014}).

\parx
Flowchart-inspired type visual programming languages provide basic and limited
capabilities for programming. The common usage for this is evaluation of the
conditionals and flow of the program.  It mainly uses arrows and boxes with simple
value.  Examples of this are Flowgorithm, Raptor, and WebML.

\parx
Dataflow type visual programming language commonly used in professional applications
moreso for designers than programmers. With this format, there is a wide selection
of available capabilites as each block represents a function or procedure which can
store and output multiple values through lines or wires. Examples of this are Unreal
Blueprint and CryEngine Flow Graph.

\parx
The Finite-state Machines (FSM) type uses basic shapes and connections only. This is
commonly used for animation and states to visualize the transition from one block to
another. Example of this is NodeCanvas.

\parx
Behavior Trees type is similar to Finite-state Machines but is more complex and allows
for multiple states to be triggered depending on the parameters that match the current
state. This is mostly used for complex animation in the game industry. Examples of
this are NodeCanvas and Craft.ai.

\parx
Event-based type of visual programming language is the simplest and most akin to
the traditional text-based programming languages. The simplest illustration to define
this is to write a programming code in text form and then assign a graphical element
or picture of each keyword. For example, the picture for the keyword "for" will be a
looping arrow. Examples of this are Construct, IFTTT, and Kodu.
