% \addcontentsline{toc}{subsection}{Ishikawa Diagram}
\flushleft
\textbf{Ishikawa Diagram}\\
\justifying

\parx
Ishikawa diagram is also called the fishbone diagram and cause-and-effect diagram in
that it is a graphical technique in the shape of a fish skeleton used to numerous causes
of a phenomenon. It is commonly used to identify and analyze causes and its complex
relation to each other that contribute to the specific problem.

A finding of a study about the use of Ishikawa diagram as an appropriate theoretical
framework for representing visually and analyzing technology of complex factors of
major improvements and innovations over the course of history and time. This graphical
representation tool presents a simple and clear the order and relation of the causes
and roots of a problem addressed by the change in technology
(\cite{coccia_2017}).
