% \fakesubsection{Persistence in a Game-Based Learning Environment: The Case of Elementary
% School Students Learning Computational Thinking}
\flushleft
\textbf{Persistence in a Game-Based Learning Environment}\\
\justifying

\parx
Persistence has been a great challenge in online learning environments.
The motivation to finish a chalenge may have a negative efffect on learning at
specific levels and hence on learning in general. Gaming and interactivity
have been suggested as important features in increasing persistence in online
learning. Persistence, the determination of learners to complete a learning
process and obtain their goals despite challenges, has been determined to be a
very useful and valuable skill when solving problems (\cite{dicerbo_2016}).
The researchers studied microlevel persistence in the context of acquiring
computational thinking, which is the thought process of solving problems through
abstraction (\cite{codemonkey_2020}).

\parx
In game-based learning environments, learners often receive immediate response
or feedback regarding the solution they have submitted. Incorrect solutions have
a different indicator as opposed to correct solutions. Instead of accepting
the solution as is if correct, there will still be a feedback to inform that
though it is correct, there can be improvements and better way to do it. To
achieve such an understanding, the learners require metacognitive knowledge and
beliefs together with metacognitive skills of monitoring and control (\cite{ishida_2002}).
