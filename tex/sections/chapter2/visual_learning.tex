% \fakesubsection{Visual Learning}
\flushleft
\textbf{Visual Learning}\\
\justifying

\parx
Information is retained more in memory through visual formats.  Visual information can
be presented in various formats such as images, diagrams, graphs, video, and
simulations. This approach in learning helps the instructors to convey their lesson
better and clearer while the students develop visual thinking skills.  This skill is the
comprehension of association of data such as concepts, theories, and ideas into
graphical elements like imagery and diagram (\cite{raiyn_2020}).

\parx
Visual learning can be improved more through the addition of interaction using visual
interactive tools. This is beneficial in many domains that require logical thinking and
skill such as programming. Interaction and visualization at the initial level of coding
motivates the interests and engagement of learners even at the young age. This approach
has been very effective for the Scratch programming environment as they adapted to
adding visualization and media content creation to programming activities which are
trends in the culture of youth. Learning through exploration and sharing to peers, this
motivated young people to focus less on direct instruction that other programming
languages provide (\cite{maloney_resnick_rusk_silverman_eastmond_2010}).

\parx
Having a physical design, blueprint, or a diagram that serves as guide for the product
or machine to be made has been the traditional method for manufacturing complex and
expensive things. The same principle applies in programming. Programmers manually input
code from their brain which can be called as mental model to task the computer into
doing a complex routine. But this is a challenge for many reasons such as other people
does not inherently have the same mental model regarding the solution and structure.
Another reason is that the level of familiarity and expertise to a particular tool
or environment used is not the same for all programmers. So intead of from one's
mental model to code, it should suffice to create and visualize the model itself
before jumping into directly generating the code. This allows for coordination between
multiple programmers as they have the same guide for the solution and concept. This
also applies for novice in programming to learn that visualization before coding is
a discipline one must come to understand and put into practice. This application of
visuals into learning and execution could be of great benefit to reduce the complexity,
effort, and time consumption (\cite{ottosson_zaslavskyi_2019}).
