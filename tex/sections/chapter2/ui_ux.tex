% \addcontentsline{toc}{subsection}{User Interface and User Experience}
\flushleft
\textbf{User Interface and User Experience}\\
\justifying

\parx
A user interface (UI) refers to a system and a user interacting with each other through
commands or techniques to operate the system, input data, and use the contents. This
ranges from systems such as computers, mobile devices, games, to application programs
and content usage. On the other hand, user experience (UX) refers to the overall
experience of the user. This includes the perception, reaction and behavior that the
user may feel and think in direct or indirect usage of the system, product, content or
services. It is a concept that is widely applied not only in software and hardware
development but also in services, products, processes, society and culture. Both UI and
UX is an interface through which a person can interact with a system or application in
a computer and communication environment, which is classified into a software and
hardware interface. Software interface is represented by the user interface while
hardware interface is categorized into a plug or an interface card connecting the
computer and its peripheral devices. UX’s has four key axes which are needs,
expectations, attributes and capabilities. Hence, it identifies the problem with the
need of the users, applies motivation and manage the expectations of the users
(\cite{joo_2017}).
