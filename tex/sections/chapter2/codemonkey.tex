% \fakesubsection{CodeMonkey}
\flushleft
\textbf{CodeMonkey}\\
\justifying

\parx
CodeMonkey is a learning environment for developing computational thinking aimed
primarily at elementary and secondary school students. This learning environment
follows the context of game-based and challenge-based learning methodology.
CodeMonkey is different than common block-based programming approach like Scratch
in that students are required to input code at the initial phase of the game.
Nevertheless, no prior programming knowledge is required. Learners need to aid
a monkey, the main character, in catching bananas while facing various challenges
and obstacles. There are multiple worlds, each with theme and set of challenges,
progressing from one world to another is only possible if the learner completes
the previous world and its challenges. Users may submit a solution to a challenge,
then the system automatically checks it and responds with immediate feedback based
on the submission. Feedback includes stars which act as rewards to further
motivate the learner into solving more challenges (\cite{codemonkey_2020}).
