% \addcontentsline{toc}{subsection}{Software Visualization}
\flushleft
\textbf{Software Visualization}\\
\justifying

\parx
Software visualization tools have different types for different use cases. The following
are broad classifications of the visualization tools: program visualization, algorithm
visualization, and visual programming.

\parx
Program visualization is used to determine the runtime behavior of a program and visualize
it for the user to see and inspect the information. This is commonly used for debugging
programs such as showing of virtual memory and CPU usage.

\parx
Algorithm visualization is used to visually show the each step in the process of running
an algorithm. An example of this is sorting algorithm wherein elements that represent
a value to be sorted are in every iteration selected, compared, and sorted. This is to
show a high level of abstraction for learning and understanding the concept of an
algorithm.

\parx
Visual Programming is similar to program visualization but is distinct enough to be set
as a different classification. Compared to other visualization type, visual programming
allows interaction with the visualization rather than the visualization can be interacted.
This is used as a mean to program visually as opposed to visually see the program.

\parx
Teaching and learning programming through visualization is a pedagogically sound
approach. The nature of a program is that the code is static but during runtime it is
dynamic. The dynamic aspect is difficult to learn at first especially for novice
programmers as they need to form a mental model of the processes involves based on
logic and set of theories (\cite{juha_sorva_2012}).
