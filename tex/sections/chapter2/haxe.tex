% \fakesubsection{Haxe}
\flushleft
\textbf{Haxe}\\
\justifying

\parx
Haxe is a high-level, Turing-complete, and packed with features programming language.
It is modernly designed and implemented that there are times it feels native Java,
sometimes JavaScript, and sometimes Python. The Haxe framework is suitable for complex
projects that can target desktop, mobile, web, and the cloud.

The unique feature of Haxe is its cross-language compilation, also called transpilation.
This language can target whatever platforms the target language is capable of. It can
run natively if targetted to C/C++, it can run in the web if targetted to JavaScript,
it can run in the mobule if targetted to Java, and more.

Haxe is also a statically-typed language which allows for the safest code to be written,
analyzed, and checked during compilation to catch minor issues. This also allows for
IDE and toolings support across a variety of softwares (\cite{coates_2018}).

Haxe being open-source allows for a population of contributors, testers, and users who
actively continue to improve the language along with popular libraries (written in other
programming languages) to be available for the Haxe ecosystem. A large repository of
packages and libraries that complement the standard library can be easily found and
integrated using the Haxe Library Manager (haxelib). To prove that Haxe can be used in
the industry and in complex and big projects, Haxe showcases successful big games,
softwares, tools, and websites (\cite{haxe_2020}).
