% \addcontentsline{toc}{subsection}{On the Design of a Generic Visual Programming
% Environment}
\flushleft
\textbf{On the Design of a Generic Visual Programming Environment}\\
\justifying

\parx
Visual programming languages are commonly embedded and coupled within
environments that are visually interactive. The identification of visual
programming language is associated with its environment. Therefore, the creation
of a visual programming language is tightly integrated with the creation of its
environment. The Requirements of a visual programming environment include
graphical elements with relationships graphically shown as each element is
connected with another element. Making an algorithm must be done graphically as
well through editing elements and creating connections between related data.
Each element contains underlying data structures that can be complex as there
are more information stored and used such as the visual representation, logical
connection, domain knowledge, and more. Parsing group of elements, or diagrams,
are difficult in that a parsing algorithm must be employed for handling such
case.

\parx
A generic visual programming environment can be represented as a group of
textual and visual specification tools. Designing a visual programming
environment requires consideration for the semantics and syntax of the language
as well as the visual interface. To handle with ease the maintenance,
modification, and reuse of a visual programming environment, modules are needed
to be specified clearly and with their interactions with one another
(\cite{da_qian_zhang_kang_zhang}).
