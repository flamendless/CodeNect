% \addcontentsline{toc}{subsection}{Gantt Chart}
\flushleft
\textbf{Gantt Chart}\\
\justifying

\parx
Gantt chart is a classic tool in project management. It is one of the most known and
widely used planning and management tool in projects in different domains. The
principles for the development and design of Gantt chart are time-focused, objective,
deterministic, analytic, accountable, and sequential
(\cite{geraldi_lechter_2012}).

\parx
Time-focus as projects have a target time for the either the completion or progress
milestone. Each task should be well coordinated in time and work as a crucial part in
project management.

\parx
Objective as projects must have ground in reality for the objectives to be met in a
realistic and feasible manner.

\parx
Deterministic as each task should be properly defined, studied, and defined. This
ensures that there should be no uncertainty in the objective and method of the tasks.

\parx
Analytical as projects are the sum of different and subset of tasks. A project must
be analyzed very well and divided properly into smaller tasks. This should take into
consideration the execution and scope of each task.

\parx
Accountable as a project gets divided into smaller parts, a project is also divided
and assigned to different person. Each person should be accountable for the progress
and completion of the task assigned.

\parx
Sequential as in the management of a project, there is always a sequence or order
required for further tasks to be started by waiting for the completion of other tasks
it depend upon. This sequence fit as a timeline analogy.
