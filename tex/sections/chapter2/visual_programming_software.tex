% \addcontentsline{toc}{subsection}{Visual Programming Software}
\flushleft
\textbf{Visual Programming Software}\\
\justifying

\parx
Visual programming is commonly built-in or provided by industry-size softwares for
allowing non-programmers to also perform complex logic and controls without the need to
learn traditional text-based programming languages. This is widely popular in the game
development field as desginers and artists can create visual effects through the use of
visual programming language.

The development of softwares that support visual programming for novice programmers such
as Scratch and Snap! results in learners learning to code without the need for grammer
correctness as needed in traditional text-based programming languages
(\cite{bau_gray_kelleher_sheldon_turbak_2017}). But Scrach and Snap! has their own
programming languages which the visual programming side exports to.  This increases the
effort and time required to transition into learning popular programming languages like
C, Python, and Java. There are environments wich allow the visual code to output in
C language but it can not execute, one needs to copy and paste the output to another
editor to run it (\cite{abe_fukawa_tanaka_2019}).

For Java, there exist numerous visual programming softwares such as Symantec Cafe,
Visual J++, and Visual Age for Java. The core feature of the softwares is to enable
end-users to manipulate elements of the interface in their natural graphical
representation. The softwares allow for editing the packages, classes, methods, and
variables of the available elements. But the following require basic understanding and
or experience already with programming, particularly in Java, to make full use of the
softwares (\cite{prokhorov_kosarev}).

AgentSheets is one of the early pioneers of the concept of visual programming. This
visual programming softwares use block-based type of graphical elements similar to
a jigsaw puzzle piece. The elements can be dragged and dropped onto another to create
composition for the logic of the program. Since most of the elements are visually
represented, this allows for comprehension as a block or group of blocks can explain
itself in terms of its purpose, control, and logic. Another key concept of visual
programming expressed in AgentSheets is easy sharing of blocks with one another
instead of looking at plain texts (\cite{repenning_2017}).
