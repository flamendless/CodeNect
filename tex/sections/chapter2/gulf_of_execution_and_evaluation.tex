% \addcontentsline{toc}{subsection}{Gulf of Execution and Evaluation}
\flushleft
\textbf{Gulf of Execution and Evaluation}\\
\justifying

\parx
This book features a study regarding the usage of things. The learning phase that occurs
during usage has two gulfs, the gulf of execution wherein the user figures out and
attempts how it works, and the gulf of evaluation wherein the user observes and
comprehends the results of usage. This understanding in the part of the user is
applicable in designing a system where the goal is to assist and improve learning.
The gulf presents cases wherein users failing to use a simple object results in
blaming one's self and users failing to learn a complex object results in forfeit in
further learning. In reality, the fault is not solely on the user, but from the
designer and the design of the object.

\parx
The study recommends that the design of the system should bridge the gulf by developing
it to be accessible and understandable relative to the expectations of the users either
through concise information or feedback per step on the behaviors
(\cite{norman_2013}).
