% \fakesubsection{HASKEU: An editor to support visual and textual programming in tandem}
\flushleft
\textbf{HASKEU: An editor to support visual and textual programming in tandem}\\
\justifying

\parx
This paper shows the effectiveness and usefulness of combining textual and
visual system where a change between both system updates both interfaces. The
application of textual and visual systems in combination with one another allows
users and learners of programming to easily develop programs. Focusing on the
visual representation but at the same time seeing the textual representation.
This enables them to understand the effect of a change in both formats. Learning
through this will make it easier for learners to transition into more advance
and complex concepts. HASKEU was developed in a research project to assist
end-user functional programming for Haskell programs (\cite{alam_bush_2016}).

\parx
Learning programming is most times a time-consuming and frustrating task.
Writing and testing programs after the skills are learned can also be a
laborious endeavor. A textual program is one where the program presents a set
of commands in textual form for function and variable definition. Textual
programming tests our analytical, logical, and verbal thinking abilities. Visual
programming tests our non-verbal thinking ability as it uses meaningful
graphical representations. Visual representations assist learning and retention
in memory and may provide an incentive to learn programming without language
barriers.

\parx
In 1990, a committee of functional programmers created a well-developed and
powerful functional programming language called Haskell. Functional languages
are based on the lambda calculus. These mean that the programs have no concept
of state. Functional programs are pure functions which take input and produce an
output. In recent years, there is an increase in the usage of functional
programming but one thing to be considered is that functional programming can be
overwhelming to learn as it is very different to other more common type of
language such as declarative or imperative programming.

\parx
HASKEU is a prototype programming and development environment for the Haskell
programming language. This programming system was developed to support both
textual and visual programming. The HCI (HumanComputer Interaction) techniques
are used extensively by the design of HASKEU. The HCI techniques include rules
of design for UI, data display, icon, and direct manipulation.
