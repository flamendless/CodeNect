% \fakesubsection{Visual Programming vs. Text-based Programming}
\flushleft
\textbf{Visual Programming vs. Text-based Programming}\\
\justifying

\parx
Programming is seen to be as the career of the future as technology continues to scale
up and improve. For this reason many countries are already promoting and implementing
programming subjects to their curriculum in primary education
(\cite{williams_alafghani_daley_gregory_rydzewski_2015}). Most of these formal subjects
use a visual programming language for teaching instead of the traditional text-based
programming language.

\parx
It would seem to be appropriate and more productive to teach
text-based language as it is the standard and the type of programming language used for
the development of softwares and applications in the industry and the real world.
However, in considering the comparison of the engagement, interest, and actual learning
of the students at the introductory level, visual programming language is more suited
and ideal as results showed that the motivation of the subjects that use visual
programming language improved (\cite{tsukamoto_2016}).

\parx
Studies also show that using a visual programming language like Scratch to teach
students programming and transitioning to real programming language (text-based) shows
a marginal improvements to their understanding of computer science concepts
(\cite{armoni_2015}). Visual-based environment as compared to textual
environment also displays positive results in terms of the interest and motivation of
the learners to pursue programming (\cite{daisuke_2017}).

\parx
Novice programmers take longer time to learn programming because of many constraints
that are needed to be learnt first such as the syntax and semantics of a programming
language. Common errors and difficulties in text-based programming languages include
type corrections, misspellings, typographical errors, and grammatical or syntax errors.
Another factor is that students whose native language is not English find it harder
to type because the keywords in most programming languages are in English. Even the
tools used, text editors or integrated development editors (IDE) are hard to operate
and use (\cite{liu_wu_dong_2010}).
