% \addcontentsline{toc}{subsection}{Difficulty in Learning Programming}
\flushleft
\textbf{Difficulty in Learning Programming}\\
\justifying

\parx
Different studies prove that many students has a poor learning in programming in the
midst of their programming education. Through observation they found out that a lot of
students are unable to read and write code effectively. There are few teachers who
claimed that their students are able to meet the standards of programming by graduation
however it was admitted that many programming graduates are still unable to program
(\cite{carter_and_jenkins}).

\parx
An average student does not make much progress in an introductory programming course
(\cite{robins_2003}). Most of the students struggled to get past in learning language
features and never had a chance to learn higher programming skills and problem-solving
strategies in programming (\cite{linn_dalbey}). Several working groups have looked into
the skill levels of the student at the end of CS1 courses in the past decades. These
frequent studies has been beneficial as they prove the mismatch between programming
education and the actual programming skill gained (\cite{mccraken_2001}). Programming
teachers believe that one must learn how to read code before learning how to write
a code. However, programming students give more attention to reading compared to writing
codes. Some says that writing a code is much easier than reading
(\cite{lister_2009}).

\parx
A study that measured students’ ability to trace through a given program’s
execution to follow-up McCraken's investigation. Multiple-choice questionnaire was given
to CS1 graduating students around the world. The questions required the students  to
predict the values of variables at given points of execution and to complete short
programs by inserting a line of code chosen from several given alternatives. The result
was disappointing across the board as it shows that many students are unable to trace
(\cite{lister_2004}).

\parx
Another study found that novices were unable to mentally trace interactions within the
system they were themselves designing (\cite{adelson_soloway}). Another study reports
that an inability to “trace code linearly” as a major theme of novice difficulties
(\cite{kaczmarczyk}). The analyses of quiz questions indicate that many students fail to
understand statement sequencing to the extent that they cannot grasp a simple three-line
swap of variable values (\cite{corney_2011}).
