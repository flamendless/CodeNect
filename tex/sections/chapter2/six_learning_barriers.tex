% \fakesubsection{Six Learning Barriers in End-User Programming Systems}
\flushleft
\textbf{Six Learning Barriers in End-User Programming Systems}\\
\justifying

\parx
The researchers identified the following aspects prone to false assumptions that include
fundamental and basic concepts in programming as barriers to learning programming.
These barriers closely related to the concept of interfaces of a programming environment
such as the constructs of the language itself and the availability of libraries,
features, and syntax that can be used to achieve desired procedures
(\cite{ko_myers_aung_2004}).

\parx
Design barriers are internal difficulties of a problem in programming. Solutions
to problems which are difficult to visualize affect the learning experience and may
lead to false assumptions and confusions.

\parx
Selection barriers occur when learners know what to do but are unable to identify which
of the available tools and features in the programming interface is to correct.

\parx
Coordination barriers are difficulties in using libraries provided by the programming
environment in compliment with another. Learner may know how to solve individual and
simple tasks but fails to combine the approaches to solve complex problems.

\parx
Use barriers are inherent to users new to the environment. The unfamiliarity to the
interface hinders its usage due to the lack of information and guide.

\parx
Understanding barriers involve the obscurity of the processes the programming interface
do that are hidden to the users. This occurs when learners fail to evaluate and
undestand the behavior of the program relative to their expectations.

\parx
Information barriers are difficulties in obtaining information about the internal
workings of the interface. This occurs when the environment provides no method for the
users to test their hypothesis regarding the behaviors of the environment.

\parx
The barriers explicitly defined are closely related to each other. The effect of having
difficulties in overcoming a barrier affects the learning of another barrier.
