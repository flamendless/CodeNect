% \addcontentsline{toc}{subsection}{Data Flow Diagram}
\flushleft
\textbf{Data Flow Diagram}\\
\justifying

\parx
A data flow diagram illustrates the processes, data stores, and external entities in
a business or other system and the connecting data flows. It is a graphical
representation of the flow of data through information system. DFD was first
proposed by Larry Constantine, the original developer of structured design in 1970s. It
is a primary artifact and is required to be created for every system in a structured
approach. It provides a different abstraction level that is useful in system designing
because of its hierarchical structure.  It shows data flow from external into the system
and shows how the data moved from one process to another. There are four symbols for
a data flow diagram: 1.) Squares or Ovals which represent external entities. It can be
a person or a group of people outside the control of the system being modeled. It shows
where information comes from and where it goes. 2.) Circles or Rounded Rectangles shapes
represent processes within the system. They show a part of the system that transforms
inputs to outputs. The name of the process  in the symbols usually explains what the
process does so that it is generally used with the verb-object phase. 3.) Arrows represents
how the data flows. It can be electronic data or physical items or both. The name of
the arrows represents the meaning of the packet that flows along. It also shows direction
to indicate whether data or items are moving out or into a process. The last symbol is
4.) Open-Ended Rectangles which represents data stores, including both electronic stores
and physical stores. Data stores might be used for accumulating data for a long or short
period of times (\cite{arwa_2016}).
