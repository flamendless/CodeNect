\flushleft
\textbf{Conclusion}
\justifying

\parx
The visual programming software is a stand-alone executable that can run
without installation in Linux and Windows platforms. The problems the study
aims to solve were identified and represented visually though Ishikawa Diagrams
and Context Diagrams. The software consists of seven modules: Input/Output
module, Visual Nodes module, Transpiler module, Filesystem module, Simulation
module, Debug module, and Assessment module. The software was evaluated using
ISO 9126 and analyzed statistically.

% TODO add result
\parx
The results of the evaluation of the study show that the visual programming software,
CodeNect, ??? understanding in the fundamentals of programming of the
students by ??? The visual programming software can serve as an prelimary tool
for teaching the basics of programming to beginners in programming. The software also
is usable and helpful to learners wanting to program but without programming subjects
or classes.

\parx
The software serves also as an alternative approach to learning programming by
providing visual elements compared to traditional text-based programming. With
that said, the software, or visual programming in itself, does not try to
compete with the power and effectiveness of using and learning text-based
programming. Students are encouraged to continue learning programming after
understanding the basics and fundamentals of programming.

% TODO add mean result
\parx
The visual programming software was evaluated using the ISO 9126 standards with the
criteria for quality including functionality, reliability, usability, maintanability,
efficiency, and portability. The software passed the criteria for evaluation and met
all the requirements and objectives having an overall mean of ???.
