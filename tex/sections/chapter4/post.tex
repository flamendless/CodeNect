\parx
For the post-assessment test, the researchers included assessment exercises
pertaining to various programming concepts in visual code code and gathered
twelve respondents with programming subjects through the use of online survey
and assessment form. The respondents were not given any time limitation.

\parx
Table \ref{table:post_result} shows the result of the post-assessment test.

\begin{longtable}[c]{lcccccc}
\caption{Post-assessment Result}
\label{table:post_result}\\ \hline
\multicolumn{1}{c}{\multirow{2}{*}{\textbf{TOPICS}}} & \multicolumn{2}{c}{\textbf{CORRECT}} & \multicolumn{2}{c}{\textbf{INCORRECT}} & \multicolumn{2}{c}{\textbf{TOTAL}} \\
\multicolumn{1}{c}{}                                 & \textbf{n}       & \textbf{\%}       & \textbf{n}        & \textbf{\%}        & \textbf{n}      & \textbf{\%}      \\ \hline
\endfirsthead
%
\endhead
%
1. Data Types                                        & 10               & 83.33\%           & 2                 & 16.67\%            & 12              & 100\%            \\
2. Unary Comparison                                  & 12               & 100\%             & 0                 & 0\%                & 12              & 100\%            \\
3. Binary Comparison                                 & 11               & 91.67\%           & 1                 & 8.33\%             & 12              & 100\%            \\
4. Ternary Comparison                                & 7                & 58.33\%           & 5                 & 41.67\%            & 12              & 100\%            \\
5. Expression Evaluation                             & 8                & 66.67\%           & 4                 & 33.33\%            & 12              & 100\%            \\
6. Array                                             & 9                & 75\%              & 3                 & 25\%               & 12              & 100\%            \\
7. Branching                                         & 12               & 100\%             & 0                 & 0\%                & 12              & 100\%            \\
8. For-Loop                                         & 10               & 83.33\%           & 2                 & 16.67\%            & 12              & 100\%           \\ \hline
\end{longtable}

\parx
Table \ref{table:post_result_scale} shows the result of the post-assessment test
(mean = 4.12, SD = 1.57) in the Likert scale.

\begin{longtable}[c]{lccc}
\caption{Post-assessment Result Evaluation}
\label{table:post_result_scale}\\ \hline
\multicolumn{1}{c}{\textbf{TOPICS}}  & \textbf{MEAN} & \textbf{\begin{tabular}[c]{@{}c@{}}STANDARD\\ DEVIATION\end{tabular}} & \textbf{INTERPRETATION} \\ \hline
\endfirsthead
%
\endhead
%
1. Data Types                        & 4.17          & 1.94                                                                  & VERY GOOD               \\
2. Unary Comparison                  & 5.00          & 0.00                                                                  & EXCELLENT               \\
3. Binary Comparison                 & 4.58          & 1.44                                                                  & EXCELLENT               \\
4. Ternary Comparison                & 2.92          & 2.57                                                                  & GOOD                    \\
5. Expression Evaluation             & 3.33          & 2.46                                                                  & GOOD                    \\
8. Array                             & 3.75          & 2.26                                                                  & VERY GOOD               \\
9. Branching                         & 5.00          & 0.00                                                                  & EXCELLENT               \\
10. For-Loop                         & 4.17          & 1.94                                                                  & VERY GOOD               \\
\multicolumn{1}{c}{\textbf{AVERAGE}} & \textbf{4.12} & \textbf{1.57}                                                         & \textbf{VERY GOOD}     \\ \hline
\end{longtable}
