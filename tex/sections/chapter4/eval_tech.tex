\textbf{Technical Evaluation Results}
\justifying
\parx
The researchers gathered ten professionals in the field of Information
Technology and Computer Science for the technical evaluation of the system on
May 24 and 25 year 2021. The technical evaluators assessed the software
according to the ISO 9126 metrics (See Appendix B) and provided feedback
through the use of Google Form Sheet.

\parx
The rating used for the evaluation: 4.21 - 5.00 as Excellent which indicates that the
software fully meets and far exceeds the most expectations and requirements.
3.41 - 4.20 as Very Good which indicates that the software fully meets and exceeds
several expectations and requirements. 2.61 - 3.40 as Good which indicates that the
system fully meets the requirements. 1.81 - 2.60 as Fair which indicates that the software
lacks in meeting the expectations and requirements. 1.00 - 1.80 as Poor, which
indicates that the software fails to meet the expectations and requirements.

% FUNCTIONALITY
\parx
The calculation of the evaluation results, Table \ref{table:tech_functionality} shows the
functionality of the software was rated "???" (mean = ???, SD = ???) with criteria
such as informative (mean = ???, SD = ???), accurate (mean = ???, SD = ???),
and interoperability (mean = ???, SD = ???).

\begin{longtable}[c]{|l|c|c|c|}
\hline
\multicolumn{1}{|c|}{\textbf{INDICATOR}}                                                                                                 & \textbf{MEAN} & \textbf{STANDARD DEVIATION} & \textbf{INTERPRETATION} \\ \hline
\endfirsthead
%
\endhead
%
\begin{tabular}[c]{@{}l@{}}1. Informative (The information is\\ clear, concise and informative to\\ the intended audience.)\end{tabular} & ???           & ???                         & ???                     \\ \hline
\begin{tabular}[c]{@{}l@{}}2. Accurate (The software\\ provides accurate and correct\\ data.)\end{tabular}                               & ???           & ???                         & ???                     \\ \hline
\begin{tabular}[c]{@{}l@{}}3. Interoperability (The modules\\ are interconnected to each other\\ and functions as a whole.)\end{tabular} & ???           & ???                         & ???                     \\ \hline
\multicolumn{1}{|c|}{\textbf{Average}}                                                                                                   & ???           & ???                         & ???                     \\ \hline
\caption{Mean Score for the Functionality of the System}
\label{table:tech_functionality}
\end{longtable}

% RELIABILITY
\parx
The calculation of the evaluation results, Table \ref{table:tech_reliability} shows the
reliability of the software was rated "???" (mean = ???, SD = ???) with criteria
such as reliable (mean = ???, SD = ???), bug free (mean = ???, SD = ???),
and standard equipment (mean = ???, SD = ???).

\begin{longtable}[c]{|l|c|c|c|}
\hline
\multicolumn{1}{|c|}{\textbf{INDICATOR}}                                                                                                                                          & \textbf{MEAN} & \textbf{STANDARD DEVIATION} & \textbf{INTERPRETATION} \\ \hline
\endfirsthead
%
\endhead
%
\begin{tabular}[c]{@{}l@{}}1. Reliable (The software is reliable\\ in normal use.)\end{tabular}                                                                                   & ???           & ???                         & ???                     \\ \hline
2. Bug free (Software is bug free.)                                                                                                                                               & ???           & ???                         & ???                     \\ \hline
\begin{tabular}[c]{@{}l@{}}3. Standard Equipment (The\\ system uses standard equipment\\ that is reliable, widely available and\\ applicable to a variety of users.)\end{tabular} & ???           & ???                         & ???                     \\ \hline
\multicolumn{1}{|c|}{\textbf{Average}}                                                                                                                                            & ???           & ???                         & ???                     \\ \hline
\caption{Mean Score for the Reliability of the System}
\label{table:tech_reliability}
\end{longtable}

% USABLITY
\parx
The calculation of the evaluation results, Table \ref{table:tech_usability} shows the
usability of the software was rated "???" (mean = ???, SD = ???) with criteria
such as understandability (mean = ???, SD = ???), operability (mean = ???, SD = ???),
and learnability (mean = ???, SD = ???).

\begin{longtable}[c]{|l|c|c|c|}
\hline
\multicolumn{1}{|c|}{\textbf{INDICATOR}}                                                                                                                     & \textbf{MEAN} & \textbf{STANDARD DEVIATION} & \textbf{INTERPRETATION} \\ \hline
\endfirsthead
%
\endhead
%
\begin{tabular}[c]{@{}l@{}}1. Understandability (The software\\ is easy to understand.)\end{tabular}                                                         & ???           & ???                         & ???                     \\ \hline
\begin{tabular}[c]{@{}l@{}}2. Operability (The software is\\ easily operated by the intended\\ user.)\end{tabular}                                           & ???           & ???                         & ???                     \\ \hline
\begin{tabular}[c]{@{}l@{}}3. Learnability (The program is\\ attractive and interesting; it\\ motivates users to continue using\\ the program.)\end{tabular} & ???           & ???                         & ???                     \\ \hline
\multicolumn{1}{|c|}{\textbf{Average}}                                                                                                                       & ???           & ???                         & ???                     \\ \hline
\caption{Mean Score for the Usability of the System}
\label{table:tech_usability}
\end{longtable}

% EFFICIENCY
\parx
The calculation of the evaluation results, Table \ref{table:tech_efficiency} shows the
effeciency of the software was rated "???" (mean = ???, SD = ???) with criteria
such as special equipment (mean = ???, SD = ???), storage (mean = ???, SD = ???),
and detection (mean = ???, SD = ???).

\begin{longtable}[c]{|l|c|c|c|}
\hline
\multicolumn{1}{|c|}{\textbf{INDICATOR}}                                                                                                                                            & \textbf{MEAN} & \textbf{STANDARD DEVIATION} & \textbf{INTERPRETATION} \\ \hline
\endfirsthead
%
\endhead
%
\begin{tabular}[c]{@{}l@{}}1. Special equipment (If the\\ program requires special\\ equipment, the requirements are\\ minimal and clearly stated by the\\ developer.)\end{tabular} & ???           & ???                         & ???                     \\ \hline
\begin{tabular}[c]{@{}l@{}}2. Storage (The program doesn’t\\ consume large amount of memory\\ that can slow down the processing\\ of the system.)\end{tabular}                      & ???           & ???                         & ???                     \\ \hline
\begin{tabular}[c]{@{}l@{}}3. Detection (The program can\\ easily identify the cause of failure\\ within the software.)\end{tabular}                                                & ???           & ???                         & ???                     \\ \hline
\multicolumn{1}{|c|}{\textbf{Average}}                                                                                                                                              & ???           & ???                         & ???                     \\ \hline
\caption{Mean Score for the Efficiency of the System}
\label{table:tech_efficiency}
\end{longtable}

% MAINTAINABILITY
\parx
The calculation of the evaluation results, Table \ref{table:tech_maintainability} shows the
maintability of the software was rated "???" (mean = ???, SD = ???) with criteria
such as function (mean = ???, SD = ???), process (mean = ???, SD = ???),
and test (mean = ???, SD = ???).

\begin{longtable}[c]{|l|c|c|c|}
\hline
\multicolumn{1}{|c|}{\textbf{INDICATOR}}                                                                                                                             & \textbf{MEAN} & \textbf{STANDARD DEVIATION} & \textbf{INTERPRETATION} \\ \hline
\endfirsthead
%
\endhead
%
\begin{tabular}[c]{@{}l@{}}1. Function (The effort required to\\ change the system functions is\\ minimal.)\end{tabular}                                             & ???           & ???                         & ???                     \\ \hline
\begin{tabular}[c]{@{}l@{}}2. Process (The program is stable\\ that if when something is changed,\\ it will not affect the processing of\\ the system.)\end{tabular} & ???           & ???                         & ???                     \\ \hline
\begin{tabular}[c]{@{}l@{}}3. Test (The effort needed to test\\ the system is minimal.)\end{tabular}                                                                 & ???           & ???                         & ???                     \\ \hline
\multicolumn{1}{|c|}{\textbf{Average}}                                                                                                                               & ???           & ???                         & ???                     \\ \hline
\caption{Mean Score for the Maintainability of the System}
\label{table:tech_maintainability}
\end{longtable}

% PORTABILITY
\parx
The calculation of the evaluation results, Table \ref{table:tech_portability} shows the
portability of the software was rated "???" (mean = ???, SD = ???) with criteria
such as installation (mean = ???, SD = ???) and adaptability (mean = ???, SD = ???).

\begin{longtable}[c]{|l|c|c|c|}
\hline
\multicolumn{1}{|c|}{\textbf{INDICATOR}}                                                                                                             & \textbf{MEAN} & \textbf{STANDARD DEVIATION} & \textbf{INTERPRETATION} \\ \hline
\endfirsthead
%
\endhead
%
\begin{tabular}[c]{@{}l@{}}1. Installation (The effort required\\ to install the system is minimal.)\end{tabular}                                    & ???           & ???                         & ???                     \\ \hline
\begin{tabular}[c]{@{}l@{}}2. Adaptability (The system has\\ the ability to adapt to new\\ specifications or operating\\ environments.)\end{tabular} & ???           & ???                         & ???                     \\ \hline
\multicolumn{1}{|c|}{\textbf{Average}}                                                                                                               & ???           & ???                         & ???                     \\ \hline
\caption{Mean Score for the Portability of the System}
\label{table:tech_portability}
\end{longtable}

% USER-FRIENDLINESS
\parx
The calculation of the evaluation results, Table
\ref{table:tech_use_friendliness} shows the user-friendliness of the software
was rated "???" (mean = ???, SD = ???) with criteria such as clarity (mean
= ???, SD = ???), objectivity of contents (mean = ???, SD = ???), and
typographical accuracy (mean = ???, SD = ???).

\begin{longtable}[c]{|l|c|c|c|}
\hline
\multicolumn{1}{|c|}{\textbf{INDICATOR}}                                                                                                                                               & \textbf{MEAN} & \textbf{STANDARD DEVIATION} & \textbf{INTERPRETATION} \\ \hline
\endfirsthead
%
\endhead
%
\begin{tabular}[c]{@{}l@{}}1. Clarity of controls\\ (Information about controls are\\ understandable and available\\ to the users.)\end{tabular}                                       & ???           & ???                         & ???                     \\ \hline
\begin{tabular}[c]{@{}l@{}}2. Objectivity of contents (The\\ language is non-discriminatory.\\ Content is free from race,\\ ethnic, gender, age and other\\ stereotypes.)\end{tabular} & ???           & ???                         & ???                     \\ \hline
\begin{tabular}[c]{@{}l@{}}3. Typographical Accuracy (The\\ content is free from spelling\\ and grammatical errors.)\end{tabular}                                                      & ???           & ???                         & ???                     \\ \hline
\multicolumn{1}{|c|}{\textbf{Average}}                                                                                                                                                 & ???           & ???                         & ???                     \\ \hline
\caption{Mean Score for the User-Friendliness of the System}
\label{table:tech_use_friendliness}
\end{longtable}

% OVERALL
\parx
The calculation of the overall evaluation results, Table
\ref{table:tech_overall} shows the software was rated "???" (mean = ???, SD
= ???) with criteria such as
functionality (mean = ???, SD = ???),
reliability of contents (mean = ???, SD = ???),
usability of contents (mean = ???, SD = ???),
effeciency of contents (mean = ???, SD = ???),
maintability of contents (mean = ???, SD = ???),
portability of contents (mean = ???, SD = ???),
and user-friendliness (mean = ???, SD = ???).

\begin{longtable}[c]{|l|c|c|c|}
\hline
\multicolumn{1}{|c|}{\textbf{INDICATOR}} & \textbf{MEAN} & \textbf{STANDARD DEVIATION} & \textbf{INTERPRETATION} \\ \hline
\endfirsthead
%
\endhead
%
Functionality                            & ???           & ???                         & ???                     \\ \hline
Reliability                              & ???           & ???                         & ???                     \\ \hline
Usability                                & ???           & ???                         & ???                     \\ \hline
Efficiency                               & ???           & ???                         & ???                     \\ \hline
Maintainability                          & ???           & ???                         & ???                     \\ \hline
Portability                              & ???           & ???                         & ???                     \\ \hline
User-friendliness                        & ???           & ???                         & ???                     \\ \hline
\multicolumn{1}{|c|}{\textbf{Average}}   & ???           & ???                         & ???                     \\ \hline
\caption{Overall Technical Evaluation Assessment of the Software}
\label{table:tech_overall}
\end{longtable}
