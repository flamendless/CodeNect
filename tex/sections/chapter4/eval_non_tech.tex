\flushleft
\textbf{Non-Technical Evaluation Results}
\justifying
\parx
The researchers gathered twelve students taking courses with programming
subjects such as Information Technology and Computer Science for the non-technical
evaluation of the system on May 24 and 25 year 2021. The non-technical evaluators
assessed the software according to the ISO 9126 metrics (See Appendix B) and
provided feedback through the use of Google Form Sheet. The evaluation process was
conducted online.

\parx
The rating used for the evaluation: 4.21 - 5.00 as Excellent which indicates that the
software fully meets and far exceeds the most expectations and requirements.
3.41 - 4.20 as Very Good which indicates that the software fully meets and exceeds
several expectations and requirements. 2.61 - 3.40 as Good which indicates that the
system fully meets the requirements. 1.81 - 2.60 as Fair which indicates that the software
lacks in meeting the expectations and requirements. 1.00 - 1.80 as Poor, which
indicates that the software fails to meet the expectations and requirements.

% FUNCTIONALITY
\parx
The calculation of the evaluation results, Table \ref{table:non_tech_functionality} shows the
functionality of the software was rated "excellent" (mean = 4.69 SD = 0.52 with criteria
such as informative (mean = 4.67 SD = 1.05) accurate (mean = 4.83 SD = 0.39)
and interoperability (mean = 4.48 SD = 0.67).

\begin{longtable}[c]{l c c c}
\caption{Mean Score for the Functionality of the System (Non-Technical)}
\label{table:non_tech_functionality} \\
\hline
\multicolumn{1}{c}{\textbf{INDICATOR}}                                                                                                 & \textbf{MEAN} & \textbf{STANDARD DEVIATION} & \textbf{INTERPRETATION} \\ \hline
\endfirsthead
%
\endhead
%
\begin{tabular}[c]{@{}l@{}}1. Informative (The information is\\ clear, concise and informative to\\ the intended audience.)\end{tabular} & 4.67           & 0.49                         & excellent                     \\
\begin{tabular}[c]{@{}l@{}}2. Accurate (The software\\ provides accurate and correct\\ data.)\end{tabular}                               & 4.83           & 0.39                         & excellent                     \\
\begin{tabular}[c]{@{}l@{}}3. Interoperability (The modules\\ are interconnected to each other\\ and functions as a whole.)\end{tabular} & 4.58           & 0.67                         & excellent                     \\
\multicolumn{1}{c}{\textbf{Average}}                                                                                                   & 4.69           & 0.52                         & excellent                     \\ \hline
\end{longtable}

% RELIABILITY
\parx
The calculation of the evaluation results, Table \ref{table:non_tech_reliability} shows the
reliability of the software was rated "very good" (mean = 4.41 SD = 0.67).

\begin{longtable}[c]{l c c c}
\caption{Mean Score for the Reliability of the System (Non-Technical)}
\label{table:non_tech_reliability} \\
\hline
\multicolumn{1}{c}{\textbf{INDICATOR}}                                                                                                                                          & \textbf{MEAN} & \textbf{STANDARD DEVIATION} & \textbf{INTERPRETATION} \\ \hline
\endfirsthead
%
\endhead
%
\begin{tabular}[c]{@{}l@{}}1. Reliable (The software is reliable\\ in normal use.)\end{tabular}                                                                                   & 4.41           & 0.67                         & excellent                     \\ \hline
\end{longtable}

% USABLITY
\parx
The calculation of the evaluation results, Table \ref{table:non_tech_usability} shows the
usability of the software was rated "very good" (mean = 4.37 SD = 0.67) with criteria
such as understandability (mean = 4.41, SD = 0.67), and learnability (mean = 4.33 SD = 0.78).

\begin{longtable}[c]{l c c c}
\caption{Mean Score for the Usability of the System (Non-Technical)}
\label{table:non_tech_usability} \\
\hline
\multicolumn{1}{c}{\textbf{INDICATOR}}                                                                                                                     & \textbf{MEAN} & \textbf{STANDARD DEVIATION} & \textbf{INTERPRETATION} \\ \hline
\endfirsthead
%
\endhead
%
\begin{tabular}[c]{@{}l@{}}1. Understandability (The software\\ is easy to understand.)\end{tabular}                                                         & 4.41           & 0.67                         & excellent                     \\
\begin{tabular}[c]{@{}l@{}}2. Learnability (The software is\\ easily operated by the intended\\ user.)\end{tabular}                                           & 4.33           & 0.78                         & excellent                     \\
\multicolumn{1}{c}{\textbf{Average}}                                                                                                                       & 4.37           & 0.73                         & excellent                     \\ \hline
\end{longtable}

% USER-FRIENDLINESS
\parx
The calculation of the evaluation results, Table
\ref{table:non_tech_use_friendliness} shows the user-friendliness of the software
was rated "excellent" (mean = 4.50 SD = 0.6) with criteria such as clarity (mean
= 4.25, SD = 0.62), objectivity of contents (mean = 4.83, SD = 0.39), and
typographical accuracy (mean = 4.3, SD = 0.82).

\begin{longtable}[c]{l c c c}
\caption{Mean Score for the User-Friendliness of the System (Non-Technical)}
\label{table:non_tech_use_friendliness} \\
\hline
\multicolumn{1}{c}{\textbf{INDICATOR}}                                                                                                                                               & \textbf{MEAN} & \textbf{STANDARD DEVIATION} & \textbf{INTERPRETATION} \\ \hline
\endfirsthead
%
\endhead
%
\begin{tabular}[c]{@{}l@{}}1. Clarity of controls\\ (Information about controls are\\ understandable and available\\ to the users.)\end{tabular}                                       & 4.25           & 0.62                         & excellent                     \\
\begin{tabular}[c]{@{}l@{}}2. Objectivity of contents (The\\ language is non-discriminatory.\\ Content is free from race,\\ ethnic, gender, age and other\\ stereotypes.)\end{tabular} & 4.83           & 0.39                         & excellent                     \\
\begin{tabular}[c]{@{}l@{}}3. Typographical Accuracy (The\\ content is free from spelling\\ and grammatical errors.)\end{tabular}                                                      & 4.3           & 0.82                         & excellent                     \\
\multicolumn{1}{c}{\textbf{Average}}                                                                                                                                                 & 4.50           & 0.60                         & excellent                     \\ \hline
\end{longtable}

% OVERALL
\parx
The calculation of the overall evaluation results, Table
\ref{table:non_tech_overall} shows the software was rated "excellent" (mean = 4.50, SD
= 0.62) with criteria such as
functionality (mean = 4.69 SD = 0.49),
reliability of contents (mean = 4.41, SD = 0.67),
usability of contents (mean = 4.37, SD = 0.73),
and user-friendliness (mean = 4.50, SD = 0.60).

\begin{longtable}[c]{l c c c}
\caption{Overall Technical Evaluation Assessment of the Software (Non-Technical)}
\label{table:non_tech_overall} \\
\hline
\multicolumn{1}{c}{\textbf{INDICATOR}} & \textbf{MEAN} & \textbf{STANDARD DEVIATION} & \textbf{INTERPRETATION} \\ \hline
\endfirsthead
%
\endhead
%
Functionality                            & 4.69           & 0.49                         & excellent                     \\
Reliability                              & 4.41           & 0.67                         & excellent                     \\
Usability                                & 4.37           & 0.73                         & excellent                     \\
User-friendliness                        & 4.50           & 0.60                         & excellent                     \\
\multicolumn{1}{c}{\textbf{Average}}   & 4.50           & 0.62                         & excellent                     \\ \hline
\end{longtable}
