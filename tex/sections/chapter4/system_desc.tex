\flushleft
\textbf{System Description}
\justifying

\parx
The study entitled "CodeNect: Visual Programming Software for Learning
Fundamentals of Programming" was developed with the purpose of aiding beginners
in learning the fundamentals of programming in the field of technology and to serve
as an intermediary software in helping beginners get familiarized with programming.

\parx
The researchers surveyed (See Appendix A) students taking
courses with programming subjects in the collegiate level. The identified
problems from the survey results show that students find it difficult to write
code and solve problems using programming. The software alleviates the
cognitive overload by providing visual elements in the form of nodes and
connections as opposed to traditional text-based style of programming.

\parx
The software was developed and completed at the College of Engineering and
Information Technology in Cavite State University - Main, Indang Campus. The
related resources such as studies and researches were gathered for valuable
information from the Internet and libraries for the benefit of further improving
this study.

\parx
The methodology used for the development of the software is the V-Model. The
researchers followed the phases of the V-Model accordingly as requirements,
system design, architecture design, module design, implementation and coding,
and testing.

\parx
The features designed and implemented as core functionalities and modules of the
software adhere to the problems and requirements identified. Each of the components
meet the needs in learning of the end-users towards programming. The seven modules
of the software are: Input/Output module, Visual Nodes module, Transpiler module,
Filesystem module, Simulation module, Debug module, and Assessment module
(See Appendix Figure \ref{fig:theoretical_framework}).

\parx
The software was developed using the C++ programming language with the usage of
GLFW and OpenGL for rendering and other multiple open-source libraries for the
Linux and Windows desktop platforms. TinyC Compiler is used for compiling and
running the transpiled C code at runtime. Adobe Photoshop and Aseprite were
used for the creation of media such as icons and logo. The evaluation method
for the system is ISO 9126 (see Appendix C).

\parx
The researchers gathered and reviewed the results of the form for feedbacks for
further evaluation of the effectiveness and functionality of the system (see \ref{}).
