\flushleft
\textbf{Pre-Assessment Test Results}
\justifying

\parx
For the pre-assessment test, the researchers composed ten questions pertaining
to various programming concepts in traditional text-based code (See Appendix A)
and gathered twelve respondents with programming subjects through the use of
online survey and assessment form. The respondents were not given any time
limitation.

\parx
Table \ref{table:pre_result} shows the result of the pre-assessment test.

\begin{longtable}[c]{lcccccc}
\caption{Pre-assessment Result}
\label{table:pre_result}\\ \hline
\multicolumn{1}{c}{\multirow{2}{*}{\textbf{TOPICS}}} & \multicolumn{2}{c}{\textbf{CORRECT}} & \multicolumn{2}{c}{\textbf{INCORRECT}} & \multicolumn{2}{c}{\textbf{TOTAL}} \\
\multicolumn{1}{c}{}                                 & \textbf{n}       & \textbf{\%}       & \textbf{n}        & \textbf{\%}        & \textbf{n}      & \textbf{\%}      \\ \hline
\endfirsthead
%
\endhead
%
1. Data Types                                        & 5                & 41.67\%           & 7                 & 58.33\%            & 12              & 100\%            \\
2. Unary Comparison                                  & 12               & 100\%             & 0                 & 0\%                & 12              & 100\%            \\
3. Binary Comparison                                 & 8                & 66.67             & 4                 & 33.33\%            & 12              & 100\%            \\
4. Ternary Comparison                                & 4                & 33.33\%           & 8                 & 66.67              & 12              & 100\%            \\
5. Expression Evaluation                             & 4                & 33.33\%           & 8                 & 66.67              & 12              & 100\%            \\
6. Array                                             & 6                & 50\%              & 6                 & 50\%               & 12              & 100\%            \\
7. Branching                                         & 7                & 58.33\%           & 5                 & 41.67\%            & 12              & 100\%            \\
8. For-Loop                                         & 3                & 25\%              & 9                 & 75\%               & 12              & 100\%           \\ \hline
\end{longtable}

\parx
Table \ref{table:pre_result_scale} shows the result of the pre-assessment test
(mean = 2.55, SD = 2.17) in the Likert scale.

\begin{longtable}[c]{lccc}
\caption{Pre-assessment Result Evaluation}
\label{table:pre_result_scale}\\ \hline
\multicolumn{1}{c}{\textbf{TOPICS}} & \textbf{MEAN} & \textbf{\begin{tabular}[c]{@{}c@{}}STANDARD\\ DEVIATION\end{tabular}} & \textbf{INTERPRETATION} \\ \hline
\endfirsthead
%
\endhead
%
1. Data Types                        & 2.08          & 2.57                                                                  & FAIR                    \\
2. Unary Comparison                  & 5.00             & 0.00                                                                     & EXCELLENT               \\
3. Binary Comparison                 & 3.33          & 2.46                                                                  & GOOD                    \\
4. Ternary Comparison                & 1.67          & 2.46                                                                  & POOR                    \\
5. Expression Evaluation             & 1.67          & 2.46                                                                  & POOR                    \\
6. Array                             & 2.50           & 2.61                                                                  & FAIR                    \\
7. Branching                         & 2.92          & 2.57                                                                  & GOOD                    \\
8. For-Loop                         & 1.25          & 2.26                                                                  & POOR                    \\
\multicolumn{1}{c}{\textbf{AVERAGE}} & \textbf{2.55} & \textbf{2.17}                                                         & \textbf{FAIR}          \\ \hline
\end{longtable}
