\flushleft
\textbf{Method}\\
\justifying

\parx
The researchers used the V-Model methodology of Software
Development Life Cycle (SDLC) for the proposed software to be developed. The
V-Model methodology is a linear development methodology that focuses and follows
a strict and incremental steps of stages. The initial phases are generally
focused on planning and designing the system, the next phases are focused on
implementation and actual programming. After that, the model will go in upwards
direction for testing and verification of the project. The development of the
software follows the timeline (See Appendix Figure \ref{fig:gantt_chart}).

\parx
The researchers conducted a survey to gather data from students, instructors,
and learners in the field of technology and under the course with programming
subjects such as Bachelor of Science in Information Technology, Bachelor of
Science in Computer Science, and Bachelor of Science in Computer Engineering as
respondents. The data gathered (See Appendix Figure \ref{dataresults1}) are
evaluated and assessed to determine the knowledge and understanding of the
respondents in regards to the fundamentals of programming, experience and
feedback on traditional text-based tools and software. Problems were identified
and the Ishikawa Diagrams were constructed (See Appendix Figure
\ref{fishbone1}, \ref{fishbone2}, and \ref{fishbone3}).

\parx
The researchers gathered twelve students with programming subjects and have
conducted pre-assessment through Google Form to evaluate the current knowledge
and understanding on programming fundamentals of the respondents who use traditional
text-based programming. The profile of the respondents are shown in Appendix E.

\flushleft
\textbf{Pre-Assessment Test Results}
\justifying

\parx
For the pre-assessment test, the researchers composed ten questions pertaining
to various programming concepts in traditional text-based code (See Appendix A)
and gathered twelve respondents with programming subjects through the use of
online survey and assessment form. The respondents were not given any time
limitation.

\parx
Table \ref{table:pre_result} shows the result of the pre-assessment test.

\begin{longtable}[c]{lcccccc}
\caption{Pre-assessment Result}
\label{table:pre_result}\\ \hline
\multicolumn{1}{c}{\multirow{2}{*}{\textbf{TOPICS}}} & \multicolumn{2}{c}{\textbf{CORRECT}} & \multicolumn{2}{c}{\textbf{INCORRECT}} & \multicolumn{2}{c}{\textbf{TOTAL}} \\
\multicolumn{1}{c}{}                                 & \textbf{n}       & \textbf{\%}       & \textbf{n}        & \textbf{\%}        & \textbf{n}      & \textbf{\%}      \\ \hline
\endfirsthead
%
\endhead
%
1. Data Types                                        & 5                & 41.67\%           & 7                 & 58.33\%            & 12              & 100\%            \\
2. Unary Comparison                                  & 12               & 100\%             & 0                 & 0\%                & 12              & 100\%            \\
3. Binary Comparison                                 & 8                & 66.67             & 4                 & 33.33\%            & 12              & 100\%            \\
4. Ternary Comparison                                & 4                & 33.33\%           & 8                 & 66.67              & 12              & 100\%            \\
5. Expression Evaluation                             & 4                & 33.33\%           & 8                 & 66.67              & 12              & 100\%            \\
6. Array                                             & 6                & 50\%              & 6                 & 50\%               & 12              & 100\%            \\
7. Branching                                         & 7                & 58.33\%           & 5                 & 41.67\%            & 12              & 100\%            \\
8. For-Loop                                         & 3                & 25\%              & 9                 & 75\%               & 12              & 100\%           \\ \hline
\end{longtable}

\parx
Table \ref{table:pre_result_scale} shows the result of the pre-assessment test
(mean = 2.55, SD = 2.17) in the Likert scale.

\begin{longtable}[c]{lccc}
\caption{Pre-assessment Result Evaluation}
\label{table:pre_result_scale}\\ \hline
\multicolumn{1}{c}{\textbf{TOPICS}} & \textbf{MEAN} & \textbf{\begin{tabular}[c]{@{}c@{}}STANDARD\\ DEVIATION\end{tabular}} & \textbf{INTERPRETATION} \\ \hline
\endfirsthead
%
\endhead
%
1. Data Types                        & 2.08          & 2.57                                                                  & FAIR                    \\
2. Unary Comparison                  & 5.00             & 0.00                                                                     & EXCELLENT               \\
3. Binary Comparison                 & 3.33          & 2.46                                                                  & GOOD                    \\
4. Ternary Comparison                & 1.67          & 2.46                                                                  & POOR                    \\
5. Expression Evaluation             & 1.67          & 2.46                                                                  & POOR                    \\
6. Array                             & 2.50           & 2.61                                                                  & FAIR                    \\
7. Branching                         & 2.92          & 2.57                                                                  & GOOD                    \\
8. For-Loop                         & 1.25          & 2.26                                                                  & POOR                    \\
\multicolumn{1}{c}{\textbf{AVERAGE}} & \textbf{2.55} & \textbf{2.17}                                                         & \textbf{FAIR}          \\ \hline
\end{longtable}


\parx
A post-assessment test is planned to be conducted during software evaluation to
assess the effect of the study on the knowledge and understanding on programming
fundamentals of the respondents.

\parx
After data gathering, the researchers assessed the gathered data and studied the
information in order to construct a context diagram representing the manual way
that the study will solve (See Appendix Figure
\ref{fig:context_diagram_manual}). The schedule of the development and the
allotted time for each is task is planned through the Gantt chart (See Appendix
Figure \ref{fig:gantt_chart}).

\parx
After which, the researchers designed and developed specifications that served
as the blueprint of the software. The libraries, packages, tools, and more are
finalized and prepared for later use (See Appendix Figure
\ref{fig:context_diagram}).

\parx
Further, the researchers identified and defined the scopes and specific
features of each module and how each is integrated along the system to work
with other modules and components to ensure that each module is decoupled and
can be tested without dependency in other module (See Appendix Figure
\ref{fig:theoretical_framework}).

\parx
The researchers started to program each module in the C++ programming
language using Vim as the primary text editor. Compiling and running the
software will be done by running a command in the terminal. The rendering
backend is the OpenGL and the GLFW for the windowing and input framework. Each
functionality of each module was tested using unit tests by running the
test every functionality that is implemented. After each module passes
the associated test, all is coupled and integrated to a single system. The
end product of this phase is the CodeNext: Visual Programming Software for
Learning Fundamentals of Programming.

\parx
After the software was developed, the researchers applied tests in the software
that was developed.  Further testings are done. Bugs, errors, and misbehaviors
were handled and fixed. This test ensured that the functionality, efficiency,
usability, portability, and reliability of the software meets the standard.

\parx
Finally, the software was subjected to acceptance testing, ISO 9126 - Product Quality
software standards was used. Several evaluators tested the software thoroughly
to specifically evaluate their experience and their learning using the standard
metrics. The technical and non-technical evaluators assessed the functionality,
reliability, usability, maintainability, portability, efficiency, and
user-friendliness of the software.

\parx
The functionalities of the software were evaluated accordingly. The
features and effectiveness of the software will be verified by ten college
students taking programming subjects as the non-technical evaluators and another ten
IT professionals as technical evaluators. Feedbacks and remarks are collected from the
evaluators for further analysis by the researchers.

\parx
The rating used for the evaluation is described in Table \ref{table:ratings_scale}:
4.21 - 5.00 as Excellent which indicates that the
software fully meets and far exceeds the most expectations and requirements.
3.41 - 4.20 as Very Good which indicates that the software fully meets and exceeds
several expectations and requirements. 2.61 - 3.40 as Good which indicates that the
system fully meets the requirements. 1.81 - 2.60 as Fair which indicates that the software
lacks in meeting the expectations and requirements. 1.00 - 1.80 as Poor, which
indicates that the software fails to meet the expectations and requirements.

\begin{longtable}[c]{|c|c|}
\hline
\multicolumn{2}{|c|}{\textbf{LIKERT SCALE}} \\ \hline
\endfirsthead
%
\endhead
%
\textbf{RANGE}   & \textbf{INTEPRETATION}   \\ \hline
4.21 - 5.00      & Excellent                \\ \hline
3.41 - 4.20      & Very Good                \\ \hline
2.61 - 3.40      & Good                     \\ \hline
1.81 - 2.60      & Fair                     \\ \hline
1.00 - 1.80      & Poor                     \\ \hline
\caption{Likert Scale for the System Evaluation}
\label{table:ratings_scale}
\end{longtable}


\parx
The post-assessment test was conducted on May 25, 2021 in response to the
pre-assessment test which was conducted during the initial stage of the study. The
questions for the post-assessment are the same question during the pre-assessment
test.

\flushleft
\textbf{Post-Assessment Test Results}
\justifying

\parx
For the post-assessment test, the researchers included assessment exercises
pertaining to various programming concepts in visual code code and gathered
twelve respondents with programming subjects through the use of online survey
and assessment form. The respondents were not given any time limitation.

\parx
Table \ref{table:post_result} shows the result of the post-assessment test.

\begin{longtable}[c]{lcccccc}
\caption{Post-assessment Result}
\label{table:post_result}\\ \hline
\multicolumn{1}{c}{\multirow{2}{*}{\textbf{TOPICS}}} & \multicolumn{2}{c}{\textbf{CORRECT}} & \multicolumn{2}{c}{\textbf{INCORRECT}} & \multicolumn{2}{c}{\textbf{TOTAL}} \\
\multicolumn{1}{c}{}                                 & \textbf{n}       & \textbf{\%}       & \textbf{n}        & \textbf{\%}        & \textbf{n}      & \textbf{\%}      \\ \hline
\endfirsthead
%
\endhead
%
1. Data Types                                        & 10               & 83.33\%           & 2                 & 16.67\%            & 12              & 100\%            \\
2. Unary Comparison                                  & 12               & 100\%             & 0                 & 0\%                & 12              & 100\%            \\
3. Binary Comparison                                 & 11               & 91.67\%           & 1                 & 8.33\%             & 12              & 100\%            \\
4. Ternary Comparison                                & 7                & 58.33\%           & 5                 & 41.67\%            & 12              & 100\%            \\
5. Expression Evaluation                             & 8                & 66.67\%           & 4                 & 33.33\%            & 12              & 100\%            \\
6. Array                                             & 9                & 75\%              & 3                 & 25\%               & 12              & 100\%            \\
7. Branching                                         & 12               & 100\%             & 0                 & 0\%                & 12              & 100\%            \\
8. For-Loop                                         & 10               & 83.33\%           & 2                 & 16.67\%            & 12              & 100\%           \\ \hline
\end{longtable}

\parx
Table \ref{table:post_result_scale} shows the result of the post-assessment test
(mean = 4.12, SD = 1.57) in the Likert scale.

\begin{longtable}[c]{lccc}
\caption{Post-assessment Result Evaluation}
\label{table:post_result_scale}\\ \hline
\multicolumn{1}{c}{\textbf{TOPICS}}  & \textbf{MEAN} & \textbf{\begin{tabular}[c]{@{}c@{}}STANDARD\\ DEVIATION\end{tabular}} & \textbf{INTERPRETATION} \\ \hline
\endfirsthead
%
\endhead
%
1. Data Types                        & 4.17          & 1.94                                                                  & VERY GOOD               \\
2. Unary Comparison                  & 5.00          & 0.00                                                                  & EXCELLENT               \\
3. Binary Comparison                 & 4.58          & 1.44                                                                  & EXCELLENT               \\
4. Ternary Comparison                & 2.92          & 2.57                                                                  & GOOD                    \\
5. Expression Evaluation             & 3.33          & 2.46                                                                  & GOOD                    \\
8. Array                             & 3.75          & 2.26                                                                  & VERY GOOD               \\
9. Branching                         & 5.00          & 0.00                                                                  & EXCELLENT               \\
10. For-Loop                         & 4.17          & 1.94                                                                  & VERY GOOD               \\
\multicolumn{1}{c}{\textbf{AVERAGE}} & \textbf{4.12} & \textbf{1.57}                                                         & \textbf{VERY GOOD}     \\ \hline
\end{longtable}

\parx
Table \ref{table:pre_post} shows the result of the pre-assessment test
(mean = 2.55, SD = 2.17) and the result of post-assessment test
(mean = 4.12, SD = 1.57). The comparison shows an increase of 61.57\% in using
the software (mean difference = -1.57, SD difference = 0.6).

\begin{longtable}[c]{lccc}
\caption{Pre and Post Assessment Result}
\label{table:pre_post}\\ \hline
\multicolumn{1}{c}{\textbf{TEST}}       & \textbf{MEAN}  & \textbf{\begin{tabular}[c]{@{}c@{}}STANDARD\\ DEVIATION\end{tabular}} & \textbf{INTERPRETATION} \\ \hline
\endfirsthead
%
\endhead
%
1. Pre-assessment                       & 2.55           & 2.17                                                                  & FAIR                    \\
2. Post-assessment                      & 4.12           & 1.57                                                                  & VERY GOOD               \\
\multicolumn{1}{c}{\textbf{DIFFERENCE}} & \textbf{-1.57} & \textbf{0.6}                                                          & \textbf{}              \\ \hline
\end{longtable}

