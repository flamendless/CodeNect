\flushleft
\textbf{Method}\\
\justifying

\parx
The researchers will use the V-Model methodology of Software
Development Life Cycle (SDLC) for the proposed software to be developed. The
V-Model methodology is a linear development methodology that focuses and follows
a strict and incremental steps of stages. The initial phases are generally
focused on planning and designing the system, the next phases are focused on
implementation and actual programming. After that, the model will go in upwards
direction for testing and verification of the project. The development of the
software follows the timeline (See Appendix Figure \ref{fig:gantt_chart}).

\parx
The V-Model figure shows the following stages:

\flushleft
\textbf{Requirements.}
\justifying
\parx
In this stage the researchers conducted a survey to gather data from
students, instructors, and learners in the field of technology and under the
course with programming subjects such as Bachelor of Science in Information
Technology, Bachelor of Science in Computer Science, and Bachelor of Science in
Computer Engineering as respondents. The data gathered
(See Appendix Figure \ref{dataresults1})
are evaluated and assessed to determine the knowledge and understanding of
the respondents in regards to the fundamentals of programming, experience and
feedback on traditional text-based tools and software. Problems are identified
and the Ishikawa Diagrams are constructed
(See Appendix Figure \ref{fishbone1}, \ref{fishbone2}, and
\ref{fishbone3}).

\flushleft
\textbf{System Design.}
\justifying
The researchers will assess the gathered data and study the information in order
to construct a context diagram representing the manual way that the study will
solve (See Appendix Figure \ref{fig:context_diagram_manual}). The schedule of the
development and the allotted time for each is task is planned through the Gantt
chart (See Appendix Figure \ref{fig:gantt_chart}).

\flushleft
\textbf{Architecture Design.}
\justifying
\parx
The researchers in this stage will design and develop specifications that will
serve as the blueprint of the software. The libraries, packages, tools, and more
are will be finalized and prepared for later use
(See Appendix Figure \ref{fig:context_diagram}).

\flushleft
\textbf{Module Design.}
\justifying
\parx
The researchers will identify and define the scopes and specific features of
each module and how will each be integrated along the system to work with other
modules and components to ensure that each module is decoupled and can be tested
without dependency in other module
(See Appendix Figure \ref{fig:theoretical_framework}).

\flushleft
\textbf{Implementation and Coding.}
\justifying
\parx
The researchers will start to program each module in the Haxe programming
language using Vim as the primary text editor. Compiling and running the software
will be done by running a command in the terminal. The rendering backend will be
the Kha and zui framework. Each functionality of each module will be tested
using unit tests by running the test every functionality that will be implemented.
After each module passes the associated test, all will be coupled and
integrated to a single system. The end product of this phase is the CodeNext:
Visual Programming Software for Learning Fundamentals of Programming.

\flushleft
\textbf{Testing.}
\justifying
\parx
The researchers will apply tests in the software that will be developed.
Further testings will be done. Bugs, errors, and misbehaviors will be handled
and fixed. This test will ensure that the functionality, efficiency,
usability, portability, and reliability of the software meets the standard.

\flushleft
\textbf{Evaluation.}
\justifying
\parx
The evaluation to be used is the ISO/IEC/IEEE 29119-4:2015. The respondents
will test the software thoroughly to specifically evaluate their experience and
their learning.
The respondents will be asked to use both traditional text-based programming and the
CodeNect: Visual Programming Software in a period of a week each. After that, the
respondents will be tasked to solve simple coding exercises. After that, the
respondents will be given a feedback form to assess their experience with both
tools and the researchers will evaluate their solutions to the coding exercises
and compare the results.
